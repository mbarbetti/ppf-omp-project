%   +------------------+
%   |   INTRODUCTION   |
%   +------------------+
\section{Introduction}\label{sec:intro}
This document reports the methodology followed and the results obtained for the final project of ``Parallel Programming Fundamentals'', a PhD course of the University of Siena given by Roberto Giorgi and Marco Procaccini, and attended from $23^{\rm{rd}}$ to $29^{\rm{th}}$ May 2022.

The aim of the project is to quantify the \emph{parallelism} achieved by a program using the OpenMP APIs~\cite{openmp} to speed up the computation of the Mandelbrot set~\cite{mandelbrot} on a single machine. The program used for this study is written in C and corresponds to a slightly modified version of the \code{omp-mandelbrot.c} script provided within the course materials \cite{classroom}. 

The elapsed time of the program has been measured for different types of \emph{scheduling} and \emph{partition sizes}, and its performance has been tested in \emph{weak} and \emph{strong scaling} conditions, as better described in Section~\ref{sec:method}. The results of the scheduling study are reported in Section~\ref{sec:res-sched}, while the speedup achieved with weak and strong scaling conditions are shown in Section~\ref{sec:res-scale}. In Section~\ref{sec:concl}, some general remarks are discussed, and an estimation of the parallelism of the program is given.